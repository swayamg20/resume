\section*{\sc Key Projects}
\vspace{-2mm}
\hrulefill
\vspace{1mm}

\excventry
{ISRO Web Based X-RAY Burst Automation}
{Feb'22 - Apr'22}
{\href{https://docs.google.com/presentation/d/10pvkgxBG-ETi1jD3QRtYjZ6shZK6lnJaGWVreLN3-80/edit?usp=sharing}{\faBook} Inter IIT Tech Meet 10.0}
{
  \begin{itemize}
    \item  Developed \textbf{web automated system} for identifying \textbf{solar} bursts, utilising the \textbf{ReactJS framework} with data segregation features
    \item Organised curve data by saving specific details in \textbf{react hooks} in \textbf{JSON} to ensure smooth management and API calls to the server
    \item Led end-to-end \textbf{product} documentation and orchestrated the development of a user journey and research, ensuring project success
    \item Won a \textbf{Silver Medal} in the \textbf{mid-prep event}, with IIT Kanpur’s contingent securing the \textbf{2nd position} in the \textbf{overall Meet}
  \end{itemize}
}

\excventry
{Journal Scrapper for Data Mining}
{Jul'22 - Dec'22}
{Supervised By- Proff. Shikhar Krishan Jha }
{
  \begin{itemize}
    \item \textbf{Automated} system for fetching \textbf{.ris/.bib} files, extracting author details programmatically, and optimizing data retrieval processes
    \item Proficiently conducted \textbf{NER (Named Entity Recognition)} to extract author entities and \textbf{DOI URLs} from scholarly articles
    \item Used Scopus \textbf{Cited By API} to implement citation counts of \textbf{input journals}, enhancing our tool’s functionality \& data accuracy
    \item Seamlessly connected \textbf{Flask app} and frontend was established through \textbf{HTTP} or other \textbf{REST-based API} calls via backend
  \end{itemize}
}

\excventry
{TA Allotment System}
{Jan'23 - May'23}
{Supervised By- Proff. Shikhar Krishan Jha }
{
  \begin{itemize}
    \item  Developed the \textbf{Teaching Assistant allotment portal}, utilizing \textbf{Node.js}, \textbf{Express}, and \textbf{MongoDB} for server side operations
    \item Fortified security with \textbf{JWT-based} \textbf{authentication}, meticulous authorization, and \textbf{email-based login} for the enrolled students
    \item Utilized \textbf{Nodemailer}, bcrypt, for secure email, password hashing, and authentication, ensuring robust data encryption \& protection
    % \item Won a \textbf{Silver Medal} in the \textbf{mid-prep event}, with IIT Kanpur’s contingent securing the \textbf{2nd position} in the overall Meet
  \end{itemize}
}


\excventry
{Security Based Honeypot System}
{May'22 - July'22}
{ \href{https://github.com/swayamg20/honeylambda}{\faGithub{}} Self Inspired Project}
% {
%   \href{https://github.com/swayamg20/honeylambda}{\faGithub{} swayamg20/honeylambda}
% }
% {May - June 2020}
{
  \begin{itemize}
  \item Employed HoneyLambda for serverless deployment using Python, responding to malicious link activity, leveraging \textbf{AWS storage}
  % \item Used an open source trivia database to fetch the questions.
  \item Used Slack API for incident response capabilities and  notifications
  \item Used \textbf{Cymon v2} for \textbf{threat} report and summaries for \textbf{analytics}
  % \item Integrated with the Messenger platform using a \textbf{webhook}.
  \item Used \textbf{Slack weboooks}, explored creation of diverse honeytokens, focusing on email-triggering variations for cybersecurity purposes
  \end{itemize}
}

\vspace{-2mm}

